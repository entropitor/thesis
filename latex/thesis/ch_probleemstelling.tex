\chapter{Probleemstelling}
\paragraph{} In deze thesis willen we onderzoeken hoe we een gecontroleerde natuurlijke taal met een eenduidige semantiek kunnen gebruiken voor kennisrepresentatie. We ontwerpen hiervoor een nieuwe taal. De bedoeling is om deze taal toegankelijk te maken voor domein experten, i.e.\ mensen zonder achtergrond in formele talen. Verder bekijken we hoe we deze taal rijk genoeg kunnen maken voor praktische problemen. Ten slotte willen we dat deze taal toepasbaar is binnen het KBS paradigma.

Voor dit laatste willen we daarom, net als RuleCNL, een onderscheid maken tussen het vocabularium (of ontologie) en de theorie (of de regels). Onder het vocabularium verstaan we een modellering van de wereld in concepten en relaties tussen deze concepten. Dit is dus een getypeerd vocabularium, in tegenstelling tot ACE en PENG. We onderzoeken welke extra voordelen dit oplevert. 

Verder onderzoeken we of we uitbreidingen op eerste-orde-logica kunnen integreren in de nieuwe CNL om zo de formele expressiviteit te verhogen. ACE en PENG hebben zich beperkt tot (een subset van) eerste-orde-logica. Het is echter algemeen gekend dat sommige problemen niet uitgedrukt kunnen worden in eerste-orde-logica. Hierdoor zijn deze talen niet altijd rijk genoeg voor praktische problemen.

De focus ligt minder op het ondersteunen van taalkundige constructies zoals anaforische referenties. Dit is reeds grondig onderzocht in ACE en PENG. Verder onderzoek zou eventueel kunnen aantonen hoe die technieken samengebracht kunnen worden met deze thesis.

In tegenstelling tot sectie \ref{sec:ASP} (\nameref{sec:ASP}) is het wel de bedoeling om een taal te construeren en dus een grammatica op te stellen. Op die manier kan de gebruiker de taal ook leren. Door de manuele constructie is de kans op overfitting ook kleiner waardoor deze aanpak ook kan schalen naar nieuwe problemen.
