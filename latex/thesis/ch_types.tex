\chapter{Types}
\label{ch:types}

Hoofdstuk~\ref{ch:framework} introduceerde een framework voor semantische analyse. Hoofdstukken~\ref{ch:lexicon} en \ref{ch:grammatica} beschreven hoe we dit framework konden gebruiken voor het vertalen van logigrammen naar logica. Op basis van deze drie hoofdstukken kunnen we nu de zinnen van een logigram omzetten naar zinnen in eerste-orde-logica. Om de computer een logigram automatisch te laten oplossen, is er ook nog nood aan een formeel vocabularium. Om dit te kunnen opstellen voegen we types toe aan het semantische framework.

\paragraph{} We beschrijven eerst het achterliggende idee en hoe we dit kunnen toepassen op logigrammen. Dan bekijken we welke aanpassingen het lexicon en de grammatica moeten ondergaan. Vervolgens leggen we uit welke informatie er nog nodig is om de types af te leiden. Ten slotte bespreken we drie problemen die we kunnen oplossen a.d.h.v. types.

\section{Het achterliggende idee}
In natuurlijke taal kan men vele zinnen vormen zonder echte betekenis die wel grammaticaal correct zijn. Bijvoorbeeld de zin ``Het gras drinkt het zingende huis'' houdt weinig steek. Idealiter zouden we zo'n zinnen willen bestempelen als foutief. Gras is namelijk niet iets dat kan drinken en een huis kan ook niet gedronken worden. We zeggen dat ze niet van het juiste type zijn.

Daarom zullen we een feature \texttt{vType}\footnote{We kiezen voor \texttt{vType} omdat sommige grammaticale categorieën reeds een feature \texttt{type} hebben. De \texttt{v} staat voor vocabularium.} toevoegen aan alle grammaticale categorieën. Ook een aantal lexicale categorieën krijgen deze feature (de meeste open categorieën alsook de voorzetsels). Het type van een naamwoordgroep komt overeen met het type van de entiteit waarnaar het verwijst. Het type van een verbale constituent is gelijk aan dat van het onderwerp. Het type van een transitief werkwoord bestaat dan weer uit een type-paar. Het eerste voor het type van het onderwerp (en dus de verbale constituent) en het andere voor het type van het lijdende voorwerp. Ook een voorzetsel heeft een paar als waarde voor \texttt{vType}. Het ene komt overeen met dat van het substantief dat getransformeerd wordt en het andere komt overeen met dat van de naamwoordgroep uit de voorzetselconstituent.

\section{Types voor logigrammen}

\paragraph{}- 1 type per woord (lijst van woord-types matching)
- Getypeerde DRS
- attributen voor countable
- derived (countable) types
- wij kozen voor inferentie, maar het even goed type checking kunnen zijn!!!
\section{Grammatica}
- noun geen vertaling meer
- introductie van types bij elke kwantor
- vrij voor de hang liggend
\section{Vragen}
- vragen voor het verder oplossen van unificatie probleem
- waarom?
- taalkundig!!!
\section{Het juiste predicaat}
- Resolve predicates on missing types
\section{Een formeel vocabularium}
- benoemen van types
- wat vertaalt naar wat? (pn -> constructed types, ...)
\section{Een correcte theorie}
- Extra axioma's based on types
