\paragraph{} Logigrammen zijn een soort van puzzels waarbij de lezer een aantal zinnen voorgeschoteld krijgt. De zinnen bevatten een aantal concepten (zoals nationaliteit, dier, kleur, ...) en een aantal voorbeelden van die concepten (Noor, Brit, kat, hond, rood, blauw, ...). Tussen elk paar van concepten is er één bijectie. De zinnen vormen beperkingen op die bijecties. Het doel van de puzzle is het achterhalen van de waarde van de bijecties. M.a.w. welke voorbeelden van de concepten bij elkaar horen. Bijvoorbeeld ``De Noor woont in het blauwe huis en heeft een kat als huisdier''.

De zinnen van logigrammen zijn redelijk gestructureerd waardoor het mogelijk is om ze automatisch om te vormen naar een meer formele representatie waar een computer mee overweg kan. Deze thesis onderzoekt hoe haalbaar deze automatische vertaling van logigrammen naar logica is.
