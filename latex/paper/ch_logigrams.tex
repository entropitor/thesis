\section{Logigrams}
Logigrams are a sort of logical puzzles. They consist of a number of sentences or clues in natural language. In the clues, a number of domains appear (e.g. nationality, color, animal, ...), each with a number of domain elements (e.g. Norwegian, Canadian, blue, red, cow, horse, ...). Between each domain there is a bijection. The goal of a logigram is to find the value of these bijections, to find which domain elements belong together. E.g. The Norwegian lives in a blue house and keeps the horse.

Logigrams are a interesting research subject because they can be considered as small specifications. Moreover, it is easy to find numerous examples of logigrams. Finally, logigrams can be expressed in fairly simple logical statements. It is these three properties that allow for easy testing of novel methods in knowledge representation.
