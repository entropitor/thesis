\documentclass[notes, dvipsnames]{beamer}

\usepackage{default}
\usepackage[dutch]{babel}
\usepackage{pgfpages}
\usepackage{pgf-pie}
\usepackage{ifthen}
\usepackage{qtree}
\usepackage[utf8]{inputenc}

\usetheme{Frankfurt}
\usecolortheme{beaver}
\setbeamertemplate{note page}{\pagecolor{yellow!25}\insertnote}
%\setbeamertemplate{note page}{\pagecolor{white}\insertnote}

\title{Automatisch vertalen van logigrammen naar een getypeerde logica}
\subtitle{Thesisverdediging}

\author{Jens Claes}
\date{28 juni 2017}

\newcommand{\seperation}{
	\vspace{1em}
	\ppause
}
\newcommand{\sseperation}{
	\vspace{1em}
}
\newcommand{\hitem}{
	\ppause
	\item
}
\newcommand{\ppause}{\onslide<+>}
\newcommand{\nnote}[1]{\note<.>{#1}}
\setbeamercovered{%
	still covered={\opaqueness<1->{0}},
	again covered={\opaqueness<1->{60}}
}
\setbeameroption{hide notes} % Only slides
%\setbeameroption{show notes on second screen=right} % Both

\newcommand{\attention}[1]{\textcolor{ForestGreen}{#1}}

%\graphicspath{ {../images/} }

\setbeamertemplate{bibliography item}{\insertbiblabel}
\begin{document}
	\frame{\titlepage}
	\section{Probleem}
	\begin{frame}{Logigrammen}
		\begin{itemize}
      \hitem Aantal concepten (nationaliteiten, dieren, kleuren, ...)
      \item Aantal voorbeelden per concept (Noor, Brit, kat, hond, ...)
      \item 1 bijectie tussen elk paar concepten
      \item Aantal hints/clues: constraints op die bijecties
			\hitem Doel: Zoek de waarde van de bijecties
			
			\seperation
			
			\item Hints in natuurlijke taal
      \item Redelijke uniforme zinstructuur
			
			\seperation

      \item Geschikt om automatisch om te zetten naar logica
		\end{itemize}
	\end{frame}
	
	\section{Doel}
	\begin{frame}{Doel Thesis}
		\begin{itemize}
			\hitem Opstellen grammatica die de (meeste) hints omvat
			\item Grammaticale zinnen mappen op een equivalent in logica
      \item Automatisch
			
			\seperation
      \item Is zo'n grammatica toepasbaar op nieuwe logigrammen?
      \item Kunnen we het vocabularium afleiden?
      \item Kunnen we types introduceren? Wat zijn we hier mee?
		\end{itemize}
	\end{frame}

  \section{Resultaten}
	\begin{frame}{De basis}
		\begin{itemize}
			\hitem Boeken van Blackburn en Bos \cite{Blackburn2005, Blackburn2006} 
      \item Een framework voor computationele semantiek
      \item Vinden van betekenissen in taal
      \item Taalkundig: alle betekenissen
			
			\seperation
      \item 4 onderdelen:
        \begin{itemize}
          \item Lexicon (vocabularium)
          \item Grammatica
          \item Lexicale semantiek
          \item Grammaticale semantiek
        \end{itemize}
      \item Lambda-calcalus
		\end{itemize}
	\end{frame}

  \subsection{}
	\begin{frame}{Lexicon}
		\begin{itemize}
      \hitem John sleeps
			\item John loves Mary
      \item A man loves Mary
      \item If a man loves Mary, every woman sleeps
      \hitem
        \begin{quote}
          \texttt{lexicalEntry(det, 'a').} \\
          \texttt{lexicalEntry(det, 'every').} \\
          \texttt{lexicalEntry(n, 'man').} \\
          \texttt{lexicalEntry(n, 'woman').} \\
          \texttt{lexicalEntry(pn, 'John').} \\
          \texttt{lexicalEntry(pn, 'Mary').} \\
          \texttt{lexicalEntry(iv, 'sleeps').} \\
          \texttt{lexicalEntry(tv, 'loves').} \\
        \end{quote}
		\end{itemize}
	\end{frame}

	\begin{frame}{Grammatica}
		\begin{itemize}
      \hitem 
        \begin{quote}
          \texttt{s --> [if], s, s.} \\
          \texttt{s --> np, vp.} \\
          \texttt{np --> pn.} \\
          \texttt{np --> det, n.} \\
          \texttt{vp --> iv.} \\
          \texttt{vp --> tv, np.} \\
        \end{quote}
      \hitem \Tree[.s [.np [.pn john ]] [.vp [.tv loves ] [.np [.pn mary ]]]]
      \ppause \Tree[.s [.np [.det a ] [.n man ]] [.vp [.tv loves ] [.np [.pn mary ]]]]
		\end{itemize}
	\end{frame}
	\begin{frame}{Grammatica}
      \Tree[.s if [.s [.np [.det a ] [.n man ]] [.vp [.tv loves ] [.np [.pn mary ]]]] [.s [.np [.det every ] [.n woman ]] [.vp [.iv sleeps ]]]]
	\end{frame}

  \subsection{}
	\begin{frame}{Bespreking}
    \begin{itemize}
      \hitem 4 onderdelen staan los van elkaar
      \begin{itemize}
        \hitem Lexicon/Vocabularium: per probleem, taalkundige categorieën
        \hitem Grammatica: contextvrij, categorial combinatorial grammar (Baral et al. \cite{Baral2008}), ...
        \hitem Lexicale semantiek: eerste-orde logica, discourse representation structures, event-based logic, ...
        \hitem Grammaticale semantiek: licht gekoppeld aan grammatica en lexicale semantiek
      \end{itemize}
    \end{itemize}
	\end{frame}

	\begin{frame}{Bijdrage tot nog toe}
    \begin{itemize}
      \hitem Toevoegen van algemene regels
        \begin{itemize}
          \item Uitbreiden grammatica (aritmetiek, extra constructies koppelwerkwoorden, ...)
          \item Uitbreiden semantiek (lexicon + grammatica)
          \item Fixen bugs semantiek (koppelwerkwoorden, negatie)
        \end{itemize}
      \hitem Ondersteuning van specifieke zinsconstructies
        \begin{itemize}
          \item ``Of A and B, one ... and the other ...''
          \item ``A was either B or C''
          \item ``Neither A nor B ...''
          \item ``A, B, C and D are all different teams''
        \end{itemize}
      \hitem Toevoegen types
    \end{itemize}
	\end{frame}

	\begin{frame}{Types}
    \begin{itemize}
      \hitem Via taalkundige ``features''
      \item Bv. $NP[num]$ meervoud of enkelvoud
      \hitem $NP[type]$: één van de concepten
      \item $TV[type]$: bijectie tussen twee concepten
      \hitem Type uniek per woord
      \begin{itemize}
        \item Automatische opdeling van entiteiten in concepten
      \end{itemize}

      \seperation
      \item Ira scored 21 points higher than the contestant from Worthington
      \hitem Ira scored 21 points higher than [the total that] the contestant from Worthington [scored]
      \begin{itemize}
          \item Injecteer functie om type-mismatch op te lossen
      \end{itemize}
    \end{itemize}
	\end{frame}
			
	\section{Referenties}
	\begin{frame}[allowframebreaks]{Referenties}
		\bibliographystyle{plain}
		\bibliography{../thesis/referenties}
	\end{frame}
	
\end{document}
