\documentclass[notes, dvipsnames]{beamer}

\usepackage{default}
\usepackage{pgfpages}
\usepackage{pgf-pie}
\usepackage{ifthen}
\usepackage{qtree}

\usetheme{Szeged}
\usecolortheme{dolphin}
\usecolortheme{beaver}
\setbeamertemplate{note page}{\pagecolor{yellow!25}\insertnote}
%\setbeamertemplate{note page}{\pagecolor{white}\insertnote}

\title{Automatische vertaling van logigrammen naar logica}
\subtitle{Tussentijdse presentatie}

\author{Jens Claes}
\date{27 maart 2017}

\newcommand{\seperation}{
	\vspace{1em}
	\ppause
}
\newcommand{\sseperation}{
	\vspace{1em}
}
\newcommand{\hitem}{
	\ppause
	\item
}
\newcommand{\ppause}{\onslide<+>}
\newcommand{\nnote}[1]{\note<.>{#1}}
\setbeamercovered{%
	still covered={\opaqueness<1->{0}},
	again covered={\opaqueness<1->{60}}
}
\setbeameroption{hide notes} % Only slides
%\setbeameroption{show notes on second screen=right} % Both

\newcommand{\sentence}[3]{
  \alt<.>{
    #1 \\
    \textcolor{Blue}{#2} \\
    \textcolor{ForestGreen}{#3} \\
  }{
    #1 \\
    #2 \\
    #3 \\
  }
}
\newcommand{\badsentence}[4]{
  \alt<.>{
    #1 \\
    \textcolor{Blue}{#2} \\
    \textcolor{Red}{#3} \\
    \textcolor{ForestGreen}{#4} \\
  }{
    #1 \\
    #2 \\
    #3 \\
    #4 \\
  }
}

\newcommand{\attention}[1]{\textcolor{ForestGreen}{#1}}

%\graphicspath{ {../images/} }

\setbeamertemplate{bibliography item}{\insertbiblabel}
\begin{document}
	\frame{\titlepage}
	\section{Probleem}
	\begin{frame}{Logigrammen}
		\begin{itemize}
      \hitem Aantal concepten (nationaliteiten, dieren, kleuren, ...)
      \item Aantal voorbeelden per concept (Noor, Brit, kat, hond, ...)
      \item 1 bijectie tussen elk paar concepten
      \item Aantal hints/clues: constraints op die bijecties
			\hitem Doel: achterhaal de bijecties
			
			\seperation
			
			\item Hints in natuurlijke taal
      \item Redelijke uniforme zinstructuur
			
			\seperation

      \item Geschikt om automatisch om te zetten naar logica
		\end{itemize}
	\end{frame}
	
	\section{Doel}
	\begin{frame}{Doel Thesis}
		\begin{itemize}
			\hitem Opstellen grammatica die de (meeste) hints omvat
			\item Grammaticale zinnen mappen op een equivalent in logica
      \item Automatisch
			
			\seperation
      \item Kunnen we vocabularium afleiden?
      \item Kunnen we types introduceren? Wat zijn we hier mee?
		\end{itemize}
	\end{frame}

  \section{Resultaten}
	\begin{frame}{De basis}
		\begin{itemize}
			\hitem Boeken van Blackburn en Bos \cite{BlackburnBosBook1, BlackburnBosBook2} 
      \item Een framework om alle betekenissen te vinden
      \item Uit de computationele taalkunde
			
			\seperation
      \item 4 onderdelen:
        \begin{itemize}
          \item Grammatica
          \item Lexicon (vocabularium)
          \item Semantiek grammatica
          \item Semantiek lexicon
        \end{itemize}
      \item Lambda-calcalus
		\end{itemize}
	\end{frame}

  \subsection{4 onderdelen}
	\begin{frame}{Lexicon}
		\begin{itemize}
      \hitem John sleeps
			\item John loves Mary
      \item A man loves Mary
      \item If a man loves Mary, every woman sleeps
      \hitem
        \begin{quote}
          \texttt{lexicalEntry(det, 'a').} \\
          \texttt{lexicalEntry(det, 'every').} \\
          \texttt{lexicalEntry(n, 'man').} \\
          \texttt{lexicalEntry(n, 'woman').} \\
          \texttt{lexicalEntry(pn, 'John').} \\
          \texttt{lexicalEntry(pn, 'Mary').} \\
          \texttt{lexicalEntry(iv, 'sleeps').} \\
          \texttt{lexicalEntry(tv, 'loves').} \\
        \end{quote}
		\end{itemize}
	\end{frame}

	\begin{frame}{Grammatica}
		\begin{itemize}
      \hitem 
        \begin{quote}
          \texttt{s --> [if], s, s.} \\
          \texttt{s --> np, vp.} \\
          \texttt{np --> pn.} \\
          \texttt{np --> det, n.} \\
          \texttt{vp --> iv.} \\
          \texttt{vp --> tv, np.} \\
        \end{quote}
      \hitem \Tree[.s [.np [.pn john ]] [.vp [.tv loves ] [.np [.pn mary ]]]]
      \ppause \Tree[.s [.np [.det a ] [.n man ]] [.vp [.tv loves ] [.np [.pn mary ]]]]
		\end{itemize}
	\end{frame}
	\begin{frame}{Grammatica}
      \Tree[.s if [.s [.np [.det a ] [.n man ]] [.vp [.tv loves ] [.np [.pn mary ]]]] [.s [.np [.det every ] [.n woman ]] [.vp [.iv sleeps ]]]]
	\end{frame}

	\begin{frame}{Lexicale semantiek}
		\begin{itemize}
      \hitem PN(Symbol): $\lambda V.V[Symbol]$
      \hitem Mary: $\lambda V.V[Mary]$

      \seperation
      \item iv(Symbol): $\lambda N.N[\lambda X.Symbol(X)]$
      \hitem sleeps: $\lambda N.N[\lambda X.sleeps(X)]$

      \seperation
      \item tv(Symbol): $\lambda N1.\lambda N2.N2[\lambda X2.N1[\lambda X1.Symbol(X2,X1)]]$
      \hitem loves: $\lambda N1.\lambda N2.N2[\lambda X2.N1[\lambda X1.loves(X2,X1)]]$
		\end{itemize}
	\end{frame}

	\begin{frame}{Grammaticale semantiek}
		\begin{itemize}
      \hitem \texttt{np --> pn.}
      \item $np^{sem} = pn^{sem}$

      \hitem \texttt{vp --> iv.}
      \item $vp^{sem} = iv^{sem}$

      \seperation
      \item \texttt{s --> np, vp.}
      \item $s^{sem} = vp^{sem}[np^{sem}]$

      \seperation
      \item \texttt{np --> det, n.}
      \item $np^{sem} = det^{sem}[n^{sem}]$
		\end{itemize}
	\end{frame}

  \subsection{Een voorbeeld}
	\begin{frame}{Semantiek: een voorbeeld}
    \begin{itemize}
      \hitem John sleeps
      \hitem \Tree[.s:vp^{sem}[np^{sem}] [.np:pn^{sem} [.pn:john^{sem} john ]] [.vp:iv^{sem} [.iv:sleeps^{sem} sleeps ]]]
    \end{itemize}
	\end{frame}
	\begin{frame}{Semantiek: een voorbeeld}
      \Tree[.s:vp^{sem}[np^{sem}] [.np:pn^{sem} [.pn:\attention{$\lambda P.P[John]$} john ]] [.vp:iv^{sem} [.iv:\attention{$\lambda N1.N1[\lambda X.sleeps(X)]$} sleeps ]]]
	\end{frame}
	\begin{frame}{Semantiek: een voorbeeld}
      \Tree[.s:vp^{sem}[np^{sem}] [.np:\attention{$\lambda P.P[John]$} [.pn:$\lambda P.P[John]$ john ]] [.vp:\attention{$\lambda N1.N1[\lambda X.sleeps(X)]$} [.iv:$\lambda N1.N1[\lambda X.sleeps(X)]$ sleeps ]]]
	\end{frame}
	\begin{frame}{Semantiek: een voorbeeld}
    \begin{itemize}
      \hitem \Tree[.s:\attention{$\{\lambda N1.N1[\lambda X.sleeps(X)]\}[\lambda P.P[John]]$} [.np:$\lambda P.P[John]$ [.pn:$\lambda P.P[John]$ john ]] [.vp:$\lambda N1.N1[\lambda X.sleeps(X)]$ [.iv:$\lambda N1.N1[\lambda X.sleeps(X)]$ sleeps ]]]
      \hitem $\{\lambda P.P[John]\}[\lambda X.sleeps(X)]$
      \hitem $\{\lambda X.sleeps(X)\}[John]$
      \hitem $sleeps(John)$
    \end{itemize}
	\end{frame}

	\begin{frame}{Semantiek: een moeilijker voorbeeld}
    \begin{itemize}
      \hitem John loves Mary
      \hitem \Tree[.s:vp^{sem}[np^{sem}] [.np:pn^{sem} [.pn:john^{sem} john ]] [.vp:tv^{sem}[np^{sem}] [.tv:loves^{sem} loves ] [.np:pn^{sem} [.pn:mary^{sem} mary ]]]]
    \end{itemize}
	\end{frame}
	\begin{frame}{Semantiek: een moeilijker voorbeeld}
    \begin{itemize}
      \hitem \Tree[.s:vp^{sem}[np^{sem}] [.np:pn^{sem} [.pn:\attention{$\lambda P.P[John]$} john ]] [.vp:tv^{sem}[np^{sem}] [.tv:\attention{$loves^{sem}$} loves ] [.np:pn^{sem} [.pn:\attention{$\lambda P.P[Mary]}$ mary ]]]]
      \item \attention{$loves^{sem} = \lambda N1.\lambda N2.N2[\lambda X2.N1[\lambda X1.loves(X2,X1)]]$}
    \end{itemize}
	\end{frame}
	\begin{frame}{Semantiek: een moeilijker voorbeeld}
    \begin{itemize}
      \hitem \Tree[.s:vp^{sem}[np^{sem}] [.np:pn^{sem} [.pn:$\lambda P.P[John]$ john ]] [.vp:tv^{sem}[np^{sem}] [.tv:$loves^{sem}$ loves ] [.np:\attention{$\lambda P.P[Mary]$} [.pn:$\lambda P.P[Mary]$ mary ]]]]
      \item $loves^{sem} = \lambda N1.\lambda N2.N2[\lambda X2.N1[\lambda X1.loves(X2,X1)]]$
    \end{itemize}
	\end{frame}
	\begin{frame}{Semantiek: een moeilijker voorbeeld}
    \begin{itemize}
      \hitem \Tree[.s:vp^{sem}[np^{sem}] [.np:\attention{$\lambda P.P[John]$} [.pn:$\lambda P.P[John]$ john ]] [.vp:\attention{$vp^{sem}$} [.tv:$loves^{sem}$ loves ] [.np:$\lambda P.P[Mary]$ [.pn:$\lambda P.P[Mary]$ mary ]]]]
      \item $loves^{sem} = \lambda N1.\lambda N2.N2[\lambda X2.N1[\lambda X1.loves(X2,X1)]]$
      \item \attention{$vp^{sem} = tv^{sem}[np^{sem}] = \{\lambda N1.\lambda N2.N2[\lambda X2.N1[\lambda X1.loves(X2,X1)]]\}[\lambda P.P[Mary]]$}
    \end{itemize}
	\end{frame}
	\begin{frame}{Semantiek: een moeilijker voorbeeld}
    \begin{itemize}
      \hitem $vp^{sem} = \{\lambda N1.\lambda N2.N2[\lambda X2.N1[\lambda X1.loves(X2,X1)]]\}[\lambda P.P[Mary]]$
      \hitem $vp^{sem} = \lambda N2.N2[\lambda X2.\{\lambda P.P[Mary]\}[\lambda X1.loves(X2,X1)]]$
      \hitem $vp^{sem} = \lambda N2.N2[\lambda X2.\{\lambda X1.loves(X2,X1)\}[Mary]]$
      \hitem $vp^{sem} = \lambda N2.N2[\lambda X2.loves(X2,Mary)]$
    \end{itemize}
	\end{frame}
	\begin{frame}{Semantiek: een moeilijker voorbeeld}
    \begin{itemize}
      \hitem \Tree[.s:\attention{$s^{sem}$} [.np:$\lambda P.P[John]$ [.pn:$\lambda P.P[John]$ john ]] [.vp:vp^{sem} [.tv:$loves^{sem}$ loves ] [.np:$\lambda P.P[Mary]$ [.pn:$\lambda P.P[Mary]$ mary ]]]]
      \item $vp^{sem} = \lambda N2.N2[\lambda X2.loves(X2,Mary)]$
      \item \attention{$s^{sem} = vp^{sem}[np^{sem}] = loves(John, Mary)$}
    \end{itemize}
	\end{frame}

  \section{Planning}
  \begin{frame}{Planning}
			\begin{itemize}
        \hitem Blok- en examenperiode: niets
        \hitem Februari
          \begin{itemize}
            \item Vocabularium formaliseren
            \item Vocabularium gebruiken bij parsen theorie
            \item Natuurlijke taal interface voor vocabularium
          \end{itemize}
        \hitem Maart
          \begin{itemize}
            \item Expressiviteit formele taal verhogen
            \item Bv. Definities, aritmetiek, aggregaten, ...
          \end{itemize}
        \hitem April
          \begin{itemize}
            \item Schrijven thesis
            \item Taal uittesten op grote probleem (expressiviteit, leesbaarheid, ...)
          \end{itemize}
        \hitem Mei
          \begin{itemize}
            \item Schrijven thesis
          \end{itemize}
			\end{itemize}
  \end{frame}
			
	\section{Referenties}
	\begin{frame}[allowframebreaks]{Referenties}
		\bibliographystyle{plain}
		\bibliography{presentatie}
	\end{frame}
	
\end{document}
